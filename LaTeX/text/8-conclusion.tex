% -----------------------------------------------------------------------------
% Conclusion
% -----------------------------------------------------------------------------
\chapter{Conclusion}
The work presented in this thesis has shown that with some basic knowledge of acoustics the implementation of a spatial active noise control system is possible. For a given setup the results are promising but the dependency on the noise source position leaves room for improvement. Fortunately, even though the field of active noise control has been around for a while, the implementation in 3-dimensional space is very much up to date. Several alternative methods and algorithms have been proposed in recent time to aim for a more realistic implementation. One paper achieves a similar degree of noise canceling using multiple circular microphone arrays which would be easier to realise then a spherical microphone array as used in this work \cite{}. While methods based on harmonic expansions require a simple error microphone geometry, spatial ANC methods based on kernel interpolation allows for arbitrary microphone placement \cite{}. One field that is yet untapped is the possible use of deep learning concepts for the improvement of the microphones and loudspeakers positions.
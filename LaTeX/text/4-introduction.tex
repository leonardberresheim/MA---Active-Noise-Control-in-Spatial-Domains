% -----------------------------------------------------------------------------
% Introduction
% -----------------------------------------------------------------------------
\chapter{Introduction}
\label{chap:introduction}

In principle, there are two ways of reducing unwanted noise; passively or actively. Passive noise control applies absorbing materials such as the sound deadening mats used in cars or headphone cushions in order to absorb the impinging noise. In general for passive noise canceling the effectiveness rises with the frequency of the noise source, making it less practicable for a lower frequency range.\cite{Chen2017} However much of the noise produced by human endeavours lingers in those low frequencies notably the tumult caused by car traffic or aircraft engines. Active noise canceling (ANC) on the other hand, which performs better at frequencies of 1 kHz and lower though is subject to limitations for higher frequencies \cite{Kaymak2006}, introduces a canceling "anti-noise" sound field via an appropriate array of secondary sources\cite{Kuo1999}, utilising the fact that when adding a wave to the same wave phase shifted by 180° the positive and the negative parts of the wave cancel each other out.\\

ANC has already found it's way into the consumer market notably in performing external noise reduction for the carrier of noise canceling headphones. The application for this technology becomes more challenging when aiming to encapsulate multiple users into the quiet zones i.e. enlarging this quiet zone. Exemplary application for \textit{spatial ANC} is noise control in automobiles or aircraft cabins \cite{Zhang2019}.\\

In \cite{Zhang2019} methods for \textit{spacial ANC} are introduced and their performance is discussed. However the authors of this paper are very modest in elucidating any fundamentals of acoustics as it seems obvious to any ANC expert but necessary to fathom the workings of ANC.\\ 
The purpose of this thesis is to replicate parts of the results furnished by the aforementioned paper and thereby gouge through the foundation in order to serve as an introduction into the field of \textit{spacial ANC}.\\

This thesis will elaborate the basics and describe the solution approach to modeling an ANC system using the wave-domain least square method in a 3-dimensional reverberating \textit{shoe-box} room for a single frequency monopole sound source.


\textit{Nota bene}\\
The ANC system described in this paper was solely implemented in a simulated environment (Matlab). Therefore some effects existing in a physical system are left out such as the secondary-path transfer function between the secondary source and the error microphone which includes the digital-to-analog (D/A) converter, reconstruction filter, power amplifier, loudspeaker, acoustic path from loudspeaker to error microphone, error microphone, preamplifier, antialiasing filter, and analog-to- digital (A/D) converter.\cite{Kuo1999}



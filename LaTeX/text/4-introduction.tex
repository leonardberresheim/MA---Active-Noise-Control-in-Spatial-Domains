% -----------------------------------------------------------------------------
% Introduction
% -----------------------------------------------------------------------------
\chapter{Introduction}
\label{chap:introduction}

In principle there's two ways of reducing unwanted noise, either passively or actively. Passive noise control applies absorbing materials such as the sound deadening mats used in cars or headphone cushions to absorb the impinging noise. In general for passive noise canceling the effectiveness rises with the frequency of the noise source, making it less practicable for a lower frequency range.\cite{Chen2017} However much of the noise produced by human endeavours lingers in those low frequencies notably the tumult caused by car traffic or aircraft engines. Active noise canceling on the other hand, which performs better at frequencies of 1 kHz and lower, though is subject to limitations for higher frequencies \

\section{Intention}
    \begin{itemize}
        \item Simulate an active noise control system. 
    \begin{itemize}
        \item Simulate the dispersion of an acoustical source in a 3D space.
        \item Model a reference/microphone.
        \item Simulate a simple ANC-(feedback) system (1 secondary source, 1 error microphone) using existing algorithms.
        \item Simulate an advanced ANC system using existing algorithms.
    \end{itemize}
        \item Research the possible use of newer/more complex controlling algorithms i.e. genetic algorithms or Convolutional Neural Networks for ANC - optimisation.
\end{itemize}


The active noise control system described in this paper was solely implemented in an simulated environment. Therefore some effects existing in a physical system are left out such as the secondary-path transfer function between the secondary source and the error microphone which includes the digital-to-analog (D/A) converter, reconstruction filter, power amplifier, loudspeaker,acoustic path from loudspeaker to error microphone, error microphone, preamplifier, antialiasing filter, and analog-to- digital (A/D) converter.\cite{Kuo1999}


